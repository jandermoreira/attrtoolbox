%! Author = Jander Moreira (moreira.jander@gmail.com)
%! Date = 27/04/2025

\documentclass[a4paper, 11pt]{article}
\usepackage[T1]{fontenc}
\usepackage[utf8]{inputenc}

\usepackage[presets]{packdoc}
\PDSetElement{Macro}{no group index, index remark = {}}
\PDSetElement{Option}{no group index, index remark = {}}

\title{
    The \PackageName{attrtoolbox} package\\
    \small \textcolor{red}{To be} V.\ATVersion\\
    \normalsize\url{https://github.com/jandermoreira/attrtoolbox}
}
\author{Jander Moreira\\\small\texttt{moreira.jander@gmail.com}}
\date{2025/04/27}

\usepackage{attrtoolbox}

\begin{document}
\maketitle
\tableofcontents
\PDPrintChanges

\PDNewVersion{0.1}{2025/04/27}
\PDAddChange{0.1}{
    description = Initial version.,
    no box, no page,
}


\section{Introduction}
This is a general package to handle keys and values in an easy way. One may think this as declaring variables in a programming language.


\section{Usage}
This package can be used by loading it in the preamble.

\begin{Macro*}{usepackage}{\OArg{options}\PArg{attrtoolbox}}{}
\end{Macro*}

This is the only available package option.

\begin{Optiondef}{readonly}{\texttt{true} | \texttt{false}}{Initially: \texttt{false}}
    When set to \texttt{true} any attempt to change the value of an attribute will result in error. Otherwise the value can be changed freely.
\end{Optiondef}

%! parser = off
\begin{PDListing}
    \usepackage[readonly]{attrtoolbox}
\end{PDListing}
%! parser = on


\section{Commands}

\subsection{Attributes}

\begin{Macrodef}{ATAttributeSet}{\OArg{class}\MArg{attribute}\MArg{value}}{}
    This sets an attribute \Argument{attribute} to \Argument{value}. The \Argument{class} is optional.

    See \MacroRef{ATClassSet}.
\end{Macrodef}

\begin{Macrodef}{ATAttributeSetFrom}{\OArg{class}\MArg{attribute}\MArg{macro}}{}
    This sets an attribute \Argument{attribute} to the contents \Argument{macro} (one expansion). The \Argument{class} is optional.

    See \MacroRef{ATClassSet}.
\end{Macrodef}

\begin{Macrodef}{ATAttributeGet}{\OArg{class}\MArg{attribute}}{}
    This expands the value of the attribute \Argument{attribute} in class \Argument{class}.
\end{Macrodef}

\begin{PDExample}
    \ATAttributeSet{name}{Ann}
    Hi, \ATAttributeGet{name}!

    \ATAttributeSet{name}{Jhon}  % new content
    Hi, \ATAttributeGet{name}!
\end{PDExample}

\begin{PDExample}
    \ATAttributeSet{equation}{$E = mc^2$}
    \ATAttributeGetTo{\content}{equation}
    \edef\formula{\ATAttributeGet{equation}}
    The content: \meaning\content\par
    The formula: \meaning\formula\par
    The formula expanded: \expandafter\meaning\formula
\end{PDExample}

\begin{Macrodef}{ATAttributeGetTo}{\OArg{class}\MArg{macro}\MArg{attribute}}{}
    This copies the content of \Argument{attribute} in class \Argument{class} to \Argument{macro}.
\end{Macrodef}



Attributes can be used with classes. When \MacroRef{ATAttributeGet} is used specifying a class name, it is equivalent to \MacroRef{ATClassGet}. All the same, a class attribute can be set with \MacroRef{ATAttributeGet}.

\begin{PDExample}
    \ATClassSet{computer}{
        processor = Intel,
        ram = 8GiB,
        hd = 1TB,
    }
    Processor: \ATAttributeGet[computer]{processor}\par
    Main memory (current): \ATAttributeGet[computer]{ram}\par
    \ATAttributeSet[computer]{ram}{16GiB}
    Main memory (future upgrade): \ATClassGet{computer}{ram}\par
    Storage: \ATAttributeGet[computer]{hd}\par
\end{PDExample}

\begin{Macrodef}{ATAttributeIfExist}{\OArg{class}\MArg{attribute}\MArg{code-if-true}\MArg{code-if-false}}{}
    Executes \Argument{code-if-true} if the \Argument{attribute} exists, otherwise execute \Argument{code-if-false}.
\end{Macrodef}

\begin{PDExample}
    \ATClassSet{someone}{
        forname = Jhon,
        surname = Doe,
    }
    Name: \ATClassGet{someone}{forname}\ATAttributeIfExist [someone]{middlename}{~\ATClassGet{someone}{middlename}}{} \ATClassGet{someone}{surname}.

    \ATClassSet{someone}{middlename = K.}
    Name: \ATClassGet{someone}{forname}\ATAttributeIfExist [someone]{middlename}{~\ATClassGet{someone}{middlename}}{} \ATClassGet{someone}{surname}.
\end{PDExample}

\begin{PDExample}
    \ATAttributeSet{title}{Mr.}
    Dear \ATAttributeIfEqual{title}{Mr.}{Sir}{Madam}, \ldots
\end{PDExample}

\subsection{Classes}
A class is a collection of attributes and their values.

\begin{Macrodef}{ATClassSet}{\MArg{class}\MArg{list-of-attributes}}{}
    This macro is a shorthand to set several attributes in a \Argument{class}. The \Argument{list-of-attributes} is an \mbox{\texttt{attribute~=~value}} comma-separated list.

    This command replaces a sequecence of \MacroRef{ATAttributeSet} calls for a single \Argument{class}.
\end{Macrodef}

\begin{Macrodef}{ATClassSetFrom}{\MArg{class}\MArg{macro}}{}
    This macro is a shorthand to set several attributes in a \Argument{class}. The \Argument{macro} is expected to hold a comma-separated list.

    This command replaces a sequecence of \MacroRef{ATAttributeSet} calls for a single \Argument{class}.
\end{Macrodef}

\begin{Macrodef}{ATClassGet}{\MArg{class}\MArg{attribute}}{}
    This macro expands the value of \Argument{attribute} from \Argument{class}.

    Note that \PDInline{\ATClassGet{myclass}{myattr}} has the same functionality of \PDInline{\ATAttributeGet[myclass]{myattr}}. (See \MacroRef{ATAttributeGet}.)
\end{Macrodef}

\begin{PDExample}
    \ATClassSet{person}{
        forname = Susan,
        surname = Smith,
        age = 33,
    }
    \ATClassGet{person}{forname} \ATClassGet{person}{surname} is \ATClassGet{person}{age} years old.
\end{PDExample}


\section{Other tools}

\subsection{Lists}

\begin{Macrodef}{ATListCreate}{\MArg{name}}{}
    New empty list
\end{Macrodef}

\begin{Macrodef}{ATListAppendItems}{\MArg{name}\MArg{list-of-items}}{}
    Add items to list
\end{Macrodef}

% \begin{PDExample}
%     \ATListCreate{people}
%     \ATListAppendItems{people}{Jane, Sean, Sophie}
%     \ATListAppendItems{people}{Margareth, Jack}
%     \begin{itemize}
%         \ATListForEach{people}{\name}{\item \name}
%     \end{itemize}
% \end{PDExample}

\begin{Macrodef}{ATListAppendItemsFrom}{\MArg{name}\MArg{macro}}{}
    Get items from macro and add them to list
\end{Macrodef}

\begin{Macrodef}{ATListForEach}{\MArg{list-name}\MArg{macro}\MArg{commands}}{}
    foreach \Argument{macro} in \Argument{list-name} do \Argument{commands}

    When specified, the \Argument{separators} must be a four-token argument with \PArg{\Argument{sep-between-two}}, \PArg{\Argument{sep-between-more-than-two}}, \PArg{\Argument{sep-between-final-two}}, and \PArg{\Argument{after-last}}. If the list has two items then \Argument{sep-between-two} is placed between them; if more than two items are present, then \Argument{sep-between-more-than-two} is placed between each item except the last, where \Argument{sep-between-final-two} is used. Finally, \Argument{after-last} (which is not a separator \textit{per se}) is placed after the last item. All separators are empty by default.

    There is a counter for each loop which can be accessed by \MacroDef{ATCounter}.
\end{Macrodef}

\begin{PDExample}
    \ATListCreate{cities}
    \ATListAppendItems{cities}{Sao Carlos}
    \def\MyCities{Paris, London, Rio de Janeiro}
    \ATListAppendItemsFrom{cities}{\MyCities}
    \begin{itemize}
        \ATListForEach{cities}{\city}{\item \city}
    \end{itemize}
\end{PDExample}

\begin{PDExample}
    \ATListCreate{cities}
    \ATListAppendItems{cities}{Sao Carlos}
    My cities: \ATListForEach{cities}{\city}[{ and }{, }{, and }{!}]{\textit{\city}}

    \ATListAppendItems{cities}{Paris}
    My cities: \ATListForEach{cities}{\city}[{ and }{, }{, and }{!}]{\textit{\city}}

    \ATListAppendItems{cities}{London, Rio de Janeiro}
    My cities: \ATListForEach{cities}{\city}[{ and }{, }{, and }{!}]{\textit{\city}}
\end{PDExample}

\begin{Macrodef}{ATForEach}{\MArg{macro}\MArg{csv-list}\OArg{separators}\MArg{commands}}{}
    for \Argument{macro} in \Argument{csv-list} do \Argument{commands}

    The list is created on the fly then discarded.
\end{Macrodef}

\begin{Macrodef}{ATForEachFrom}{\MArg{macro}\MArg{macro-with-list}\OArg{separators}\MArg{commands}}{}
    for \Argument{macro} in content of \Argument{macro-with-list} do \Argument{commands}.

    The list is created on the fly then discarded.
\end{Macrodef}



\begin{PDExample}
    My list: \ATForEach{\fruit}{apple, orange, lemon, peach}[{ and }{, }{, and }{.}]{\fruit}

    My list: \ATForEach{\fruit}{apple, orange, lemon, peach}[{ and }{, }{, and }{.}]{(\ATCounter)~\fruit}
\end{PDExample}

\printindex


\end{document}
