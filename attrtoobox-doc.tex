%! Author = Jander Moreira (moreira.jander@gmail.com)
%! Date = 27/04/2025

\documentclass[a4paper, 11pt]{article}
\usepackage[T1]{fontenc}
\usepackage[utf8]{inputenc}
\usepackage[brazilian]{babel}

\usepackage[presets]{packdoc}

\title{
    The \PackageName{attrtoolbox} package\\
    \small V.\ATVersion\\
    \normalsize\url{https://github.com/jandermoreira/attrtoolbox}
}
\author{Jander Moreira\\\small\texttt{moreira.jander@gmail.com}}
\date{2025/04/27}

\usepackage{attrtoolbox}

\begin{document}
\maketitle
\PDPrintChanges

\PDNewVersion{0.1}{2025/04/27}
\PDAddChange{0.1}{
    description = Initial version.,
    no box, no page,
}


\section{Introduction}
This is a general package to handle keys and values in an easy way. One may think this as declaring variables in a programming language.


\section{Usage}
This package can be used by loading it in the preamble.

\begin{Macro*}{usepackage}{\OArg{options}\PArg{attrtoolbox}}{}
\end{Macro*}

This is the only available package option.

\begin{Optiondef}{readonly}{\texttt{true} | \texttt{false}}{Initially: \texttt{false}}
    When set to \texttt{true} any attempt to change the value of an attribute will result in error. Otherwise the value can be changed freely.
\end{Optiondef}

\begin{PDListing}
    \usepackage[readonly]{attrtoolbox}
\end{PDListing}


\section{Commands}

\subsection{Key-value pairs}

\begin{Macrodef}{ATAttributeSetValue}{\MArg{class}\MArg{attribute}\MArg{value}}{}
    This sets an attribute named \Argument{attribute} in a class \Argument{class} to \Argument{value}.
\end{Macrodef}

\begin{Macrodef}{ATAttributeGetValue}{\MArg{class}\MArg{attribute}}{}
    This expands the value of the attribute \Argument{attribute} in class \Argument{class}. Both \Argument{class} and \Argument{attribute} can be empty.
\end{Macrodef}

\begin{PDExample}
    \ATAttributeSetValue{}{}{Hi}

    \ATAttributeSetValue{person}{name}{Ann}
    \ATAttributeGetValue{}{}, \ATAttributeGetValue{person}{name}!

    \ATAttributeSetValue{person}{name}{Jhon}
    \ATAttributeGetValue{}{}, \ATAttributeGetValue{person}{name}!


\end{PDExample}


\begin{Macrodef}{ATAttributeSetValues}{\MArg{class}\MArg{list-of-attribute}}{}
    This macro is a shorthand to set several attributes in a \Argument{class}. The \Argument{list-of-attributes} is an \mbox{\texttt{attribute~=~value}} comma-separated list.
\end{Macrodef}

\begin{PDExample}
    \ATAttributeSetValues{person 1}{
        name = Albert,
        surname = Smith,
        age = 34,
    }
\end{PDExample}

\subsection{Lists}

\begin{Macrodef}{ATListCreate}{\MArg{name}}{}
    New empty list
\end{Macrodef}

\begin{Macrodef}{ATListAppendItems}{\MArg{name}\MArg{list-of-items}}{}
    Add items to list
\end{Macrodef}

\begin{PDExample}
    \ATListCreate{people}
    \ATListAppendItems{people}{Jane, Sean, Sophie}
    \ATListAppendItems{people}{Margareth, Jack}
    \begin{itemize}
        \ATListIterate{people}{\name}{\item \name}
    \end{itemize}
\end{PDExample}

\begin{Macrodef}{ATListAppendItemsFrom}{\MArg{name}\MArg{macro}}{}
    Get items from macro and add them to list
\end{Macrodef}

\begin{PDExample}
    \ATListCreate{cities}
    \ATListAppendItems{cities}{Sao Carlos}
    \def\MyCities{Paris, London, Rio de Janeiro}
    \ATListAppendItemsFrom{cities}{\MyCities}
    List: \ATListIterate{cities}{\city}{\fbox{\city\strut}}
\end{PDExample}


\printindex

\end{document}
